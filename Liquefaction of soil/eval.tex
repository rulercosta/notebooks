\chapter{Evaluating Liquefaction}

\section{Introduction}
To assess the liquefaction potential of soil, the following are required:
\begin{enumerate}
    \item Field tests
    \item Laboratory analysis
    \item Empirical correlations
    \item Engineering judgment
\end{enumerate}

\section{Process}

\subsection{Soil Sampling and Testing}
\begin{itemize}
    \item \textbf{Field Investigation:} Collect soil samples from the site using methods like Standard Penetration Test (SPT), Cone Penetration Test (CPT), or borehole sampling.
    \item \textbf{Laboratory Testing:} Determine soil properties such as grain size distribution, relative density, water content, and shear strength.
\end{itemize}

\subsection{Determination of Soil Properties}
\begin{itemize}
    \item \textbf{Soil Type:} Identify if the soil is sandy, silty, or clayey, as sandy soils are more prone to liquefaction.
    \item \textbf{Saturation Level:} Evaluate the groundwater level and degree of saturation, as fully saturated soils are more likely to liquefy.
    \item \textbf{Density and Relative Density:} Measure the soil density and relative density, which influence the soil's resistance to liquefaction.
\end{itemize}

\subsection{Seismic Parameters}
\begin{itemize}
    \item \textbf{Peak Ground Acceleration (PGA):} Estimate the maximum ground acceleration during an earthquake, usually from seismic hazard maps or site-specific studies.
    \item \textbf{Magnitude of Earthquake:} Consider the magnitude of the design earthquake for the region.
\end{itemize}

\subsection{Cyclic Stress Ratio}
Calculate the CSR using the following equation:
\begin{equation}
    \mathrm{CSR} = 0.65 \left( \frac{a_{max}}{g} \right) \left( \frac{\sigma_v}{\sigma_v^{'}} \right) r_d
\end{equation}

\begin{itemize}
    \item where:
    \begin{itemize}
        \item \(a_{max}\) is the peak ground acceleration (if not given, assume it to be equal to the seismic zone factor \(Z\)).
        \item \(g\) is the acceleration due to gravity.
        \item \(\sigma_v\) is the total vertical stress.
        \item \(\sigma_v'\) is the effective vertical stress.
        \item \(r_d\) is the depth reduction factor.
    \end{itemize}
\end{itemize}

\subsection{Cyclic Resistance Ratio}
Determine the CRR based on empirical correlations from SPT or CPT data. The CRR represents the soil's resistance to liquefaction under cyclic loading.

\subsection{Factor of Safety}
Calculate the FS against liquefaction using:
\begin{equation}
    \mathrm{FS} = \frac{\mathrm{CRR}}{\mathrm{CSR}}
\end{equation}
\(\mathrm{FS} \leq 1\) suggests a potential for liquefaction.
