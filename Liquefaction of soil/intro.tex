\chapter{Soil Liquefaction Evaluation}

\section{Introduction}
\subsection{What is Liquefaction?}
Soil behaves as a liquid (viscous) during an earthquake, posing significant structural risks.

\subsection{Why Evaluate Liquefaction Potential?}
Evaluating the liquefaction potential eases the determination of the chance that a soil sample will liquefy under pressure/load, enabling more informed decision-making.

\section{Objectives}
\begin{itemize}
    \item Determine the intensity of the earthquake the soil can withstand before liquefaction occurs.
    \item \textit{Earthquake Resistance}: Ensure the structure can withstand the "threshold earthquake force" without liquefaction.
    \item Find characteristic components of the soil at depths up to 30m, with 1.5m samples.
\end{itemize}
\clearpage
\newpage
\section{Factors Affecting Liquefaction}
Consider the following factors when evaluating liquefaction potential:
\begin{enumerate}
    \item Soil thickness
    \item Soil looseness
    \item Underground water level
    \item Void ratio
    \item Stress situation
    \item Stress date and geological age
\end{enumerate}

\section{Mitigation Measures}
If soil is susceptible to liquefaction, consider:
\begin{itemize}
    \item Soil densification
    \item Drainage improvement
    \item Soil replacement
    \item Deep foundations
\end{itemize}

\section{Mathematical Predictions}
Utilize advanced mathematical approaches to predict liquefaction potential:
\begin{itemize}
    \item Monte Carlo Approach for probabilistic modeling.
    \item Logistic Regression approach to obtain a smooth curve.
    \item Fuzzy Logic to determine parameter weightage.
    \item Feature Engineering to reduce required parameters.
\end{itemize}

\section{Deterministic Modeling}
\begin{enumerate}
    \item Develop a comprehensive understanding of soil behavior under earthquakes by incorporating multiple parameters.
    \item Use deterministic modeling to predict soil liquefaction potential based on specific input values.
    \item Consider the limitations of deterministic modeling, such as its reliance on simplified assumptions and neglect of uncertainty.
\end{enumerate}

\section{Probabilistic Modeling}
Calculate the percentage chance of soil liquefaction occurring by incorporating probabilistic approaches:
\begin{enumerate}
    \item Use Monte Carlo simulations or other techniques to generate a probability distribution for soil liquefaction potential.
    \item Consider how probabilistic modeling can account for uncertainty and variability in input parameters.
\end{enumerate}

\subsection{Advantages of Probabilistic Modeling}
\begin{enumerate}
    \item Provides a more comprehensive understanding of soil behavior by accounting for uncertainty and variability.
    \item Allows for the calculation of risk levels and probability of failure.
    \item Can be used to optimize the design and operation of structures based on probabilistic analysis.
\end{enumerate}

\section{3-D Modeling Options}
Leverage 3D modeling tools to visualize and analyze soil behavior:
\begin{enumerate}
    \item IPython (Free)
    \item OriginLabs (Proprietary)
\end{enumerate}
